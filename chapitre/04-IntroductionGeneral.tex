\chapter*{\centering \textcolor{cyan}{INTRODUCTION GÉNÉRALE}}
\markboth{\MakeUppercase{INTRODUCTION GÉNÉRALE}}{}
\renewcommand{\labelitemi}{$\bullet$}
%\addstarredchapter{INTRODUCTION GÉNÉRALES}
\addcontentsline{toc}{chapter}{INTRODUCTION GÉNÉRALE}
\adjustmtc
\thispagestyle{MyStyle}
Durant ces dernières années, l’informatique s’est imposée d’une manière très impressionnante dans les entreprises, cela est dû à son apport fructueux dans le domaine de gestion de base de données et pour assurer la bonne communication.\par

Aujourd’hui, vu l’intérêt croissant de vouloir gagner du temps, de communiquer mieux avec des moyens plus perfomante et plusieurs autres raisons, il a été question de chercher des solutions informatiques susceptibles de répondre aux besoins des utilisateurs.\par
C’est dans ce cadre que s’inscrit notre stage d’été qui consiste à réaliser une application «\textbf{Développement d’une application web de montage et diffusion du contenus vidéos}» .\par
au sein de la société Djagora. Cette application permet d’informatiser les différentes tâches relatives à la gestion et la diffusions des vidéos . Ce rapport sera réparti en 3 chapitres:\par 
\begin{itemize}
        \item Le premier chapitre, intitulé «\textbf{Présentation générale du projet}» est consacré à présenter l’environnement du projet et le cadre de sa réalisation.
    \item Le deuxième chapitre, intitulé «\textbf{spécifications des besoins} » présente la partie de spécifications des besoins de notre application .
    \item Le dernier chapitre, intitulé «\textbf{Réalisation} » présente l’architecture technique de l’application et les interfaces.
\end{itemize}
